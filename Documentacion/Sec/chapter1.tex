

\chapter{Marco Teórico: Inteligencia Artificial}

En el presente capitulo se expone la teoría sobre la inteligencia artificial, se  ideas de diversos autores sobre la inteligencia artificial. Ademas el capitulo servirá para plasmar las bases de la investigación del proyecto.

%********************************** %First Section  **************************************
\section{Concepto de Inteligencia Artificial} %Section - 1.1 

La Inteligencia Artificial (IA) es una de las ramas de las ciencias de la computación que más interés ha despertado en la actualidad, debido a su enorme campo de aplicación. La búsqueda de mecanismos que nos ayuden a comprender la inteligencia y realizar modelos y simulaciones de estos, es algo que ha motivado a muchos científicos a elegir esta área de investigación (Ponce et al., 2014).

A lo largo de la historia diferentes autores han dado definiciones para la inteligencia artificial, algunos otros no le dan conceptos dado a lo extenso que implica dicho concepto.

Algunos autores lo definen como sistemas que piensan como humanos: 

“El nuevo y excitante esfuerzo de hacer que los computadores piensen... maquinas con mentes, en el más amplio sentido 		literal.” (Haugeland, 1985).


“[La automatización de] actividades que vinculamos con procesos de pensamiento humano, actividades como la toma de decisiones, resolución de problemas, aprendizaje...” (Bellman 1978).

Otros autores la definen como sistemas que actúan como humanos.

“El arte de desarrollar maquinas con capacidad para realizar funciones que cuando son realizadas por personas requieren de inteligencia.” (Kurzweil, 1990).


“EI estudio de cómo lograr que los computadores realicen tareas que, por el momento, los humanos hacen mejor.” (Rich y Knight, 1991).


Todas estas definiciones hacen referencia a la idea de lograr un comportamiento inteligente semejante al humano; sin embargo, dentro de la IA también se estudian aquellos problemas que les resultan difíciles inclusos a los humanos. (Guerrero y López, 2001).



%********************************** %Second Section  *************************************
\section{Categorías de la Inteligencia Artificial} %Section - 1.2

A lo largo del tiempo se han determinado cuatro categorías en las cuales se enfoca la inteligencia artificial. Existen dos grupos, uno enfocado en los humanos y el otro centrado en torno a la racionalidad. 
El enfoque centrado en el comportamiento humano debe ser una ciencia empírica, que incluya hipótesis y confirmaciones mediante experimentos. El enfoque racional implica una combinación de matemáticas e ingeniería. Cada grupo al mismo tiempo ha ignorado y ha ayudado al otro. (Russell y Norving, 2004).

\newpage
\begin{table}[H]
\centering
\caption{Categorías de la inteligencia artificial (Russell y Norving, 2004).}
\begin{turn}{90}
\begin{tabular}{|l|l|}
\hline
\textbf{Sistemas que piensan como humanos.} & \textbf{Sistemas que piensan racionalmente.} \\ \hline
\begin{tabular}[c]{@{}l@{}}«El nuevo y excitante esfuerzo de hacer \\ que los computadores piensen… máquinas con\\ mentes, en el más amplio sentido literal».\\ (Haugeland, 1985).\\ \\ «{[}La automatización de{]} actividades que vinculamos \\ con procesos de pensamiento humano, actividades \\ como la toma de decisiones, resolución de problemas, \\ aprendizaje…» (Bellman, 1978).\end{tabular} & \begin{tabular}[c]{@{}l@{}}«El estudio de las facultades mentales mediante \\ el uso de modelos computacionales».\\  (Charniak y McDermott, 1985).\\ \\ «El estudio de los cálculos que hacen posible \\ percibir, razonar y actuar». (Winston, 1992).\end{tabular} \\ \hline
\textbf{Sistemas que actúan como humanos.} & \textbf{Sistemas que actúan racionalmente} \\ \hline
\begin{tabular}[c]{@{}l@{}}«El arte de desarrollar máquinas con capacidad \\ para realizar funciones que cuando son realizadas \\ por personas requieren de inteligencia». (Kurzweil, 1990).\\ \\ «El estudio de cómo lograr que los computadores\\ realicen tareas que, por el momento, los humanos\\  hacen mejor». (Rich y Knight, 1991).\end{tabular} & \begin{tabular}[c]{@{}l@{}}«La Inteligencia Computacional es el estudio\\ del diseño de agentes inteligentes». \\ (Poole et al., 1998).\\ \\ «IA… está relacionada con conductas\\ inteligentes en artefactos».\\  (Nilsson, 1998).\end{tabular} \\ \hline
\end{tabular}
\end{turn}
\end{table}

%********************************** % Third Section  *************************************
\section{Comparación entre Inteligencia Artificial e Inteligencia Natural}  %Section - 1.3 

El cerebro humano es el sistema de reconocimiento de patrones más complejo y eficiente que conocemos. Los humanos realizamos acciones tan sorprendentes como identificar a un conocido entre la multitud o reconocer de oído el solista de un concierto para violín. En el cerebro humano, las funciones cognitivas se realizan mediante la activación coordinada de unas 90.000.000.000 células nerviosas interconectadas mediante enlaces sinápticos. La activación neuronal sigue complejos procesos biofísicos que garantizan un funcionamiento robusto y adaptativo, y nos permite realizar funciones como el procesado de información sensorial, la regulación fisiológica de los órganos, el lenguaje o la abstracción matemática. (Benítez, Escudero, Kanaan y Misip, 2013).

La neurociencia actual todavía no aporta una descripción detallada sobre cómo la activación individual de las neuronas da lugar a la formación de representaciones simbólicas abstractas. Lo que sí parece claro es que en la mayoría de procesos cognitivos existe una separación de escalas entre la dinámica a nivel neuronal y la aparición de actividad mental abstracta. Esta separación de escalas supone la ruptura del vínculo existente entre el hardware (neuronas) y el software de nuestro cerebro (operaciones abstractas, estados mentales), y constituye la hipótesis de partida para que los símbolos abstractos puedan ser manipulados por sistemas artificiales que no requieran un substrato fisiológico natural. La posibilidad de manipular expresiones lógicas y esquemas abstractos mediante sistemas artificiales es la que permite la existencia de lo que conocemos como inteligencia artificial. (Benítez et al., 2013).

Una de las cuestiones de mayor relevancia y aún no resueltas de la neurociencia actual es saber si existen procesos mentales como la conciencia, la empatía o la creatividad-, que estén intrínsecamente ligados a la realidad biofísica del sistema nervioso humano y sean por tanto inaccesibles a un sistema artificial. (Benítez et al., 2013).

La inteligencia artificial (IA) es una disciplina académica relacionada con la teoría de la computación cuyo objetivo es emular algunas de las facultades intelectuales humanas en sistemas artificiales. Con inteligencia humana nos referimos típicamente a procesos de percepción sensorial (visión, audición, etc.) y a sus consiguientes procesos de reconocimiento de patrones, por lo que las aplicaciones más habituales de la IA son el tratamiento de datos y la identificación de sistemas. Eso no excluye que la IA, desde sus inicios en la década del 1960, haya resuelto problemas de carácter más abstracto como la demostración de teoremas matemáticos, la adquisición del lenguaje, el jugar al ajedrez o la traducción automática de textos. El diseño de un sistema de inteligencia artificial normalmente requiere la utilización de herramientas de disciplinas muy diferentes como el cálculo numérico, la estadística, la informática, el procesado de señales, el control automático, la robótica o la neurociencia. Por este motivo, pese a que la inteligencia artificial se considera una rama de la informática teórica, es una disciplina en la que contribuyen de forma activa numerosos científicos, técnicos y matemáticos. En algunos aspectos, además, se beneficia de investigaciones en áreas tan diversas como la psicología, la sociología o la filosofía. (Benítez et al., 2013).

Pese a que se han producido numerosos avances en el campo de la neurociencia desde el descubrimiento de la neurona por Santiago Ramón y Cajal a finales del siglo XIX, las tecnologías actuales están muy lejos de poder diseñar y fabricar sistemas artificiales de la complejidad del cerebro humano. De hecho, a día de hoy estamos lejos de reproducir de forma sintética las propiedades electroquímicas de la membrana celular de una sola neurona. Pero como hemos comentado anteriormente, la manipulación de conceptos y expresiones abstractas no está supeditada a la existencia de un sistema biológico de computación. En definitiva, un ordenador no es más que una máquina que procesa representaciones abstractas siguiendo unas reglas predefinidas. (Benítez et al., 2013).

En ocasiones, los sistemas de IA resuelven problemas de forma heurística mediante un procedimiento de ensayo y error que incorpora información relevante basada en conocimientos previos. Cuando un mismo problema puede resolverse mediante sistemas naturales (cerebro) o artificiales (computadora), los algoritmos que sigue cada implementación suelen ser completamente diferentes puesto que el conjunto de instrucciones elementales de cada sistema es también diferente. El cerebro procesa la información mediante la activación coordinada de redes de neuronas en áreas especializadas (cortex visual, cortex motor, etc.). En el sistema nervioso, los datos se transmiten y reciben codificados en variables como la frecuencia de activación de las neuronas o los intervalos en los que se generan los potenciales de acción neuronales. El elevado número de neuronas que intervienen en un proceso de computación natural hace que las fluctuaciones fisiológicas tengan un papel relevante y que los procesos computacionales se realicen de forma estadística mediante la actividad promediada en subconjuntos de neuronas. (Benítez et al., 2013).

La tabla siguiente plasma las diferencias más relevantes entre los sistemas de inteligencia natural y los de inteligencia artificial en diferentes niveles.

\newpage
\begin{table}[H]
\centering
\caption{Comparación entre inteligencia natural y artificial a diferentes niveles (Benítez et al., 2013).}
\begin{turn}{90}
\label{my-label}
\begin{tabular}{|l|l|l|}
\hline
\multicolumn{1}{|c|}{\textbf{Nivel}} & \multicolumn{1}{c|}{\textbf{Inteligencia Natural}} & \multicolumn{1}{c|}{\textbf{Inteligencia Artificial}} \\ \hline
Abstracción & Representación y manipulación de,objetos abstractos. & Representación y manipulación de,objetos abstractos. \\ \hline
Computación & Activación coordinada de áreas cerebrales. & Algoritmo / procedimiento efectivo \\ \hline
Programación & Conexiones sinápticas de plasticidad. & Secuencia de operaciones aritmético-lógicas. \\ \hline
Arquitectura & Redes excitatorias e inhibitorias. & CPU + memoria \\ \hline
Hardware & Neurona. & Transistor \\ \hline
\end{tabular}
\end{turn}
\end{table}

\section{Historia de la Inteligencia Artificial}

Desde tiempos inmemoriales, el hombre ha buscado la materialización del deseo de crear seres semejantes a él; pasando por la creación de artefactos con aspecto, movimientos y hasta comportamiento similar al que presentamos los seres humanos. El ruso Isaac Asimov (1920-1992), escritor e historiador, narraba sobre objetos y situaciones que en su tiempo eran ciencia-ficción; sin embargo, con el paso del tiempo, muchas de ellas se han ido volviendo realidad. Asimov, en su libro Runaround describió lo que el día de hoy son las tres leyes de la robótica. Su obra literaria serviría como motivación para que los científicos e ingenieros trataran de hacerla realidad. (Ponce et al., 2014).

La historia de la inteligencia a pesar de tener su tiempo, esta no cuenta con avances números como lo son otras ciencias, de acuerdo a lo planteado por Ponce (2010) y los autores citados por el mismo, a continuacion, se presenta la evolucion de la inteligencia artificial a travez del tiempo.

Se podría considerar que unos de los primeros pasos hacia la inteligencia artificial fueron dados hace mucho tiempo por Aristóteles (384-322 a.C.), cuando se dispuso a explicar y codificar ciertos estilos de razonamiento deductivo que él llamó silogismos. Otro intento sería el de Ramón Llull (d.C. 1235-1316), místico y poeta catalán, quien construyó un conjunto de ruedas llamado Ars Magna, el cual se suponía iba a ser una máquina capaz de responder todas las preguntas.

Por su parte, Martin Gardner (Gardner, 1982) atribuye a Gottfried Leibniz (1646-1716) el sueño de "un álgebra universal por el cual todos los conocimientos, incluyendo las verdades morales y metafísicas, pueden algún día ser interpuestos dentro de un sistema deductivo único”. Sin embargo, no existió un progreso sustancial hasta que George Boole (Boole, 1854) comenzó a desarrollar los fundamentos de la lógica proposicional. El objeto de Boole fue, entre otros: “recoger... algunos indicios probables sobre la naturaleza y la constitución de la mente humana”. Poco después, Gottlob Frege propuso un sistema de notación para el razonamiento mecánico y al hacerlo inventó gran parte de lo que hoy conocemos como cálculo proposicional (lógica matemática moderna) (Frege, 1879).

En 1958, John McCarthy, responsable de introducir el término “inteligencia artificial”, propuso utilizar el cálculo proposicional como un idioma para representar y utilizar el conocimiento en un sistema que denominó la “Advice Taker”. A este sistema se le tenía que decir qué hacer en vez de ser programado. Una aplicación modesta pero influyente de estas ideas fue realizada por Cordell Green en su sistema llamado QA3.

Lógicos del siglo XX, entre ellos Kart Codel, Stephen Kleene, Emil Post, Alonzo Church y Alan Turing, formalizaron y aclararon mucho de lo que podía y no podía hacerse con los sistemas de lógica y de cálculo. En fechas más recientes, científicos de la computación como Stephen Cook y Richard Karp, descubrieron clases de cálculos que, aunque parecían posibles en principio, podrían requerir cantidades totalmente impracticables de tiempo y memoria (almacenamiento).

Algunos filósofos (Lucas, 1961; Penrose, 1989; Penrose, 1994) interpretaron como una confirmación que la inteligencia humana nunca será mecanizada. Warren McCulloch y Walter Pitts escribieron teorías acerca de las relaciones entre los elementos de cálculo simple y las neuronas biológicas (McCuylloch y Pitts, 1943). Otro trabajo realizado por Frank Rosenblatt (1962) exploró el uso de redes llamadas perceptrones. Otras corrientes de trabajo, entre ellos la cibernética (Wiener 1948), la psicología cognitiva, la lingüística computacional (Chomsky, 1965) y la teoría de control adaptable, contribuyeron a la matriz intelectual de la inteligencia artificial y su desarrollo.

Gran parte del trabajo inicial de la inteligencia artificial se desarrolló en la década de 1960 y principios de los setenta en programas como General Problem Solver (GPS) de Allan Newell, Cliff Shaw y Herbert Simon (Newell, Shaw y Simon, 1959; Newell y Show 1963). Otros sistemas que influyeron son: la integración simbólica de Slagle en 1963, álgebra word por Bobrow en 1968, analogy puzzles de Evans en 1968 y control y robots móviles por Nilsson en 1984. Muchos de estos sistemas son el tema de un artículo llamado Computers and Thought (Feigenbaum y Feldman, 1963).

Hacia finales de los setenta y principios de los ochenta, algunos programas que se desarrollaron contenían mayor capacidad y conocimientos necesarios para imitar el desempeño humano de expertos en varias tareas. El primer programa que se le atribuye la demostración de la importancia de grandes cantidades de conocimiento y dominio específico es DENDRAL, un sistema de predicción de la estructura de las moléculas orgánicas que considera su fórmula química y el análisis de espectrograma de masa [Feigenbaum, Buchanan and Lederberg, 1971]. Le siguieron otros “sistemas expertos” como por ejemplo (Shortliffe, 1976; Millar, Pople y Myers, 1982) sistemas computacionales configurables (McDermott, 1982) y otros más.

En mayo 11 de 1997, un programa de IBM llamado Deep Blue derrotó al actual campeón mundial de ajedrez, Garry Kasparov. Por otra parte, Larry Roberts desarrolló uno de los primeros programas de análisis de escena. Este trabajo fue seguido por una amplia labor de máquinas de visión (visión artificial). Otros proyectos que se pueden mencionar son CYC, una de cuyas metas era recolectar e interpretar gran cantidad de información para su conocimiento. Aunque el interés en las redes neurales se estancó un poco después de los trabajos pioneros de Frank Rosenblatt en los últimos años de la década de 1950, se reanudó con energía en los años ochenta. En la actualidad hay distintas aplicaciones con la inteligencia artificial.

Softbots (Etzioni y Weld, 1994) son agentes de software que deambulan por la Internet, encontrando información que piensan será útil a sus usuarios al acceder a Internet. La presión constante para mejorar las capacidades de los robots y los agentes de software motivarán y guiarán mayores investigaciones de inteligencia artificial en los años venideros.

\section{Inteligencia Artificial Convencional y Computacional}

\subsection{Inteligencia Artificial Convencional}
Tiene que ver con métodos que actualmente se conocen como máquinas de aprendizaje, se caracteriza por el formalismo y el análisis estadístico. Algunos métodos de esta rama incluyen:

\begin{itemize}
\item Sistemas expertos: aplican capacidad de razonamiento para lograr una conclusión. Un sistema experto puede procesar una gran cantidad de información conocida y proveer conclusiones basadas en ésta.
\item Razonamiento basado en casos: es la parte de la inteligencia artificial que se ocupa del estudio de los mecanismos mentales necesarios para repetir lo que se ha hecho o vivido con anterioridad, ya sea por experiencia propia o por casos concretos recopilados en la bibliografía o en la sabiduría popular. Los diversos casos son del tipo "Si X, entonces Y" con algunas adaptaciones y críticas según las experiencias previas en el resultado de cada una de dichas reglas.
\item Comportamiento basado en Inteligencia Artificial: método modular para construir sistemas de IA manualmente.
\item Red Bayesiana: un modelo de representación del conocimiento basado en teoría de la probabilidad. (Ponce et al., 2014).
\end{itemize}

\subsection{Inteligencia Artificial Computacional}

Implica el aprendizaje interactivo. Este aprendizaje está basado en datos empíricos y está asociado con una inteligencia artificial no simbólica. Algunos métodos de esta rama incluyen:

\begin{itemize}
\item Redes neuronales: sistemas con grandes capacidades de reconocimiento de patrones.
\item Sistemas difusos: técnicas para lograr el razonamiento bajo incertidumbre. Ha sido ampliamente usada en la industria moderna y en productos de consumo masivo, como las lavadoras, asimismo es extremadamente popular en robótica porque permite la creación acertada de los sistemas dinámicos en tiempo real que pueden funcionar en ambientes complejos. Por ejemplo, es la base de la inteligencia de Sony Aibo.
\item Computación evolutiva: aplica conceptos inspirados en la biología, tales como población, mutación y supervivencia del más apto para generar soluciones sucesivamente mejores para un problema. Estos métodos a su vez se dividen en algoritmos evolutivos (ej. algoritmos genéticos) e inteligencia colectiva (ej. algoritmos hormigas). (Ponce et al., 2014).
\end{itemize}

\section{Principales Ámbitos de Aplicación de la Inteligencia Artificial}

Como ciencia la inteligencia artificial tiene diversas aplicaciones en diferentes ámbitos. Algunas mas complejas que otras, algunas áreas muy distintas a otros, que tienen en común la aplicación de la inteligencia artificial en sus respectivas áreas.

Las aplicaciones más frecuentes de la inteligencia artificial incluyen campos como la robótica, el análisis de imágenes o el tratamiento automático de textos. (Benítez et al., 2013)

\begin{table}[H]
\centering
\caption{Principales ámbitos de aplicación de los sistemas de inteligencia artificial (Benítez et al., 2013).}
\label{my-label}
\begin{tabular}{|l|l|}
\hline
\multicolumn{1}{|c|}{\textbf{Area}} & \multicolumn{1}{c|}{\textbf{Aplicaciones}} \\ \hline
Medicina & \begin{tabular}[c]{@{}l@{}}Ayuda al diagnóstico.\\ Análisis de imágenes biomédicas.\\ Procesado de señales fisiológicas.\end{tabular} \\ \hline
Ingeniería & \begin{tabular}[c]{@{}l@{}}Organización de la producción.\\ Optimización de procesos.\\ Cálculo de estructuras.\\ Planificación y logística.\\ Diagnóstico de fallos.\\ Toma de decisiones.\end{tabular} \\ \hline
Economía & \begin{tabular}[c]{@{}l@{}}Análisis financiero y bursátil.\\ Análisis de riesgos.\\ Estimación de precios en \\ productos derivados.\\ Minería de datos.\\ Marketing y fidelización. de clientes\end{tabular} \\ \hline
Biología & \begin{tabular}[c]{@{}l@{}}Análisis de estructuras biológicas.\\ Genética médica y molecular.\end{tabular} \\ \hline
Informática & \begin{tabular}[c]{@{}l@{}}Procesado de lenguaje natural.\\ Criptografía.\\ Teoría\\ de juegos.\\ Lingüística computacional.\end{tabular} \\ \hline
\begin{tabular}[c]{@{}l@{}}Robótica y \\ automática\end{tabular} & \begin{tabular}[c]{@{}l@{}}Sistemas adaptativos de rehabilitación.\\ Interfaces cerebro-computadora.\\ Sistemas de visión artificial.\\ Sistemas de navegación automática.\end{tabular} \\ \hline
\begin{tabular}[c]{@{}l@{}}Física y \\ matemáticas\end{tabular} & \begin{tabular}[c]{@{}l@{}}Demostración automática de teoremas.\\ Análisis cualitativo sistemas no-lineales.\\ Caracterización de sistemas complejos.\end{tabular} \\ \hline
\end{tabular}
\end{table}

A continuación, se describen una serie de problemas en la vida real que han sido solucionados usando inteligencia artificial, ya sea usando reconocimiento de patrones, redes neuronales, lógica difusa, y algunas otras ramas de la inteligencia artificial.

\begin{itemize}
\item[1.]Desentrelazado de señales de video con lógica difusa.

La eficacia de la lógica difusa para manejar la ambigüedad e imprecisión que aparece en numerosos problemas ha motivado en los últimos años una creciente aplicación de dichas técnicas al procesado de imágenes. Muchos de los trabajos realizados se han centrado en dos aplicaciones que son objeto de una gran demanda en la actualidad: el desentrelazado de señales de video y el incremento de resolución de imágenes. (Ponce, 2010).

\item[2.]Marcadores anatómicos de los ventrículos del corazón.

La ventriculografía es una técnica utilizada para visualizar las cavidades cardiacas. Su objetivo principal es definir el tamaño y la forma del ventrículo izquierdo, así como también visualizar la forma y la movilidad de estructuras asociadas con las válvulas del corazón. Extraer la forma ventricular ha sido uno de los principales problemas que se encuentran al aplicar las técnicas de procesamiento digital de imágenes a imágenes de las cavidades cardiacas. (Ponce, 2010).

Muchos autores coinciden en que la solución del problema de generación automática de un contorno aproximado debe estar relacionada con la aplicación de técnicas basadas en inteligencia artificial. En tal sentido, se propone el uso de las redes neurales para identificar los marcadores anatómicos necesarios para establecer una representación de un contorno inicial sobre imágenes angiográficas del ventrículo izquierdo. (Ponce, 2010).

\item[3.]Segmentación de imágenes cerebrales de resonancia magnética basada en redes neuronales.

El estudio a través de imágenes de los cambios estructurales del cerebro puede proveer información útil para el diagnóstico y el manejo clínico de pacientes con demencia. Las imágenes de resonancia magnética pueden mostrar anormalidades que no son visibles en la tomografía computarizada. Asimismo, tienen el potencial de detectar señales de anormalidad, lo que permite un diagnóstico diferencial entre la enfermedad de Alzheimer y la demencia vascular. (Ponce, 2010).

Se presenta un método de segmentación de imágenes de resonancia magnética cerebrales basada en la utilización de redes neurales utilizando algoritmos genéticos para el ajuste de los parámetros. (Ponce, 2010).

\item[4.]Optimización de sistemas para tratamiento de agua (Austria, lógica difusa).

Este caso trata de una solución lógica difusa en la producción de bioquímicos en la industria más grande de producción de penicilina en Austria. Después de extraer la penicilina a partir de los microorganismos que la generan, un sistema de tratamiento de aguas residuales procesa las biomasas sobrantes. Ahora bien, los lodos fermentados que se obtienen en el curso de este tratamiento contienen microorganismos y restos de sales nutrientes, los cuales son el material base para fertilizantes de alta calidad que se venden como coproductos de la producción de penicilina. (Ponce, 2010).

Para hacer el abono, el lodo fermentado se decanta y el agua sobrante se vaporiza. El objetivo es reducir los costos del proceso de vaporización, por lo que el proceso de decantación del agua y las sustancias para fertilizante deben ser optimizados. Así, se implementó una solución utilizando lógica difusa para dejar de operar manualmente este proceso. (Ponce, 2010).
\end{itemize}

Ahora bien, después de revisar algunas de las aplicaciones de la inteligencia artificial en problemas de la vida real, es decir, dando soluciones a problemas reales. Revisaremos igual, algunas de las aplicaciones más recientes de la inteligencia artificial. Cabe resaltar que la lista que se mostrara a continuación contiene las tecnologías de inteligencia artificial que destacan actualmente y causaran un mayor impacto en el futuro.

\begin{itemize}
\item[1.]Generación de lenguaje natural.

Es un sub-campo de la inteligencia artificial que consiste en crear texto a partir de datos obtenidos. Esto permite que las computadoras puedan comunicar ideas con gran precisión y exactitud. (Maynez, 2017, parr. 7).

Algunos proveedores que brindan este servicio son: Attivio, Automated Insights, Cambridge Semantics, Digital Reasoning, Lucidworks, Narrative Science, SAS, Yseop. (Maynez, 2017, parr. 9).

\item[2.]Reconocimiento de voz.

Siri no es el único agente que te entiende. Cada vez más sistemas incorporan la transcripción y transformación del lenguaje humano a formatos útiles para las computadoras. (Maynez, 2017, parr. 10).

Actualmente se implementa en sistemas interactivos voice response (reconocimiento de voz) y en aplicaciones móviles. (Maynez, 2017, parr. 11).

\item[3.]Agentes virtuales.

Un agente virtual es una computadora o programa capaz de interactuar con humanos. El ejemplo más común de esta tecnología son los chatbots. (Maynez, 2017, parr. 13).

Actualmente se utiliza en servicio y atención al cliente y para la administración de las Smart homes (casas inteligentes). (Maynez, 2017, parr. 14).

\item[4.]Plataformas machine learning.
 
El aprendizaje automático o aprendizaje de máquinas (en inglés machine learning) es el sub-campo de las ciencias de la computación y una rama de la inteligencia artificial cuyo objetivo es desarrollar técnicas que permitan a las computadoras aprender. (Maynez, 2017, parr. 17).

Adext es el primer y único BMaaS (Budget Management as a Service) que aplica Inteligencia Artificial y Machine Learning a la publicidad digital para encontrar la mejor audiencia o grupo demográfico para cualquier anuncio. (Maynez, 2017, parr. 22).

\item[5.]Hardware optimizado con IA.

El hardware tiene que comenzar a ser más amigable con las tecnologías de inteligencia artificial, y esto concibe la creación de unidades procesadoras de gráficos y dispositivos específicamente diseñados y estructurados para ejecutar tareas orientadas a la inteligencia artificial. (Maynez, 2017, parr. 26).

\end{itemize}

\section{Conclusiones}
Por medio de la investigación que se acaba de presentar, se puede concluir que la inteligencia artificial no es algo nuevo, que lleva años de desarrollo y que su evolución desde que comenzó ha sido constante. También, que los deferentes autores tiene definiciones propias basadas en su perspectiva de la inteligencia artificial y que de acuerdo a estas definiciones existen cuatro categorías en las que se puede clasificar la inteligencia artificial.

Por otro lado, la inteligencia artificial, a diferencia de lo que se cree, tiene muchas mas aplicaciones que solo a la robótia, durante la investigación se plantearon también ejemplos donde la inteligencia artificial fue aplicada en otras áreas, y se establecieron también las nuevas tendencias de aplicaciones de la inteligencia artificial.

Igual se realizo la comparación entre la inteligencia natural y artificial, donde se expusieron las diferencias entre ambas en diferentes niveles, y los casos para los cuales una es mas optima que la otra. También se realizo una descripción de la inteligencia artificial computacional y convencional en la cual se detalla para que sirven cada una y sus diferentes implementaciones.
\chapter{INTRODUCCIÓN}
\section{Planteamiento del Probléma}
La ortografía son un conjunto de reglas o normas hechas para regular la escritura, tanto de la composición de las palabras, como del correcto uso de los signos de puntuación.

La ortografía con el paso del tiempo ha ido perdiendo para las nuevas generaciones la importancia y relevancia que tenía en un principio. Como Fernández-Rufete  (2015) menciona “En la actualidad, existe un sentir común entre los docentes: el cada vez más grave problema de la ortografía y las dificultades para atajarlo” (sección de Introducción, párr. 1).

En el mismo documento Fernández-Rufete (2015) no explica que parte del problema de la ortografía se debe a la inclusión e influencia de los dispositivos electrónicos para la escritura, quitando estos el aprecio por la corrección en la misma actividad (escribir), ya que estos se requieren que sea todo instantáneo e improvisado. Por lo cual el uso de una correcta ortografía es menospreciado para las nuevas generaciones, a pesar de que la ortografía y su correcta aplicación es una necesidad, como Gómez (2009) establece que:

\tab “Escribir sin faltas de ortografía, separando bien las palabras, poniendo la letra y la tilde adecuadas a cada caso, las tildes, etc., es siempre señal de pulcritud mental. Además, las personas que escriben con faltas de ortografía, con desaliño en la separación de las palabras, sin tildes, etc., aparecen como incultas o semi-analfabetas. Por otra parte, hoy por hoy, la escritura correcta supone prestigio social y un buen aval para encontrar un trabajo digno.” (p. 16).
\section{Objetivo General}

\section{Objetivos Especificos}
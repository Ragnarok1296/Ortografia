% aqui definimos el encabezado y pie de pagina de la pagina inicial de un capitulo.
%\fancypagestyle{fancy}{
%\fancyhead[L]{Referencias}
%\fancyhead[C]{}
%\fancyhead[R]{}
%\fancyfoot[L]{}
%\fancyfoot[C]{}
%\fancyfoot[R]{}
%\renewcommand{\headrulewidth}{0.5pt}
%\renewcommand{\footrulewidth}{0.5pt}
%}


\begin{thebibliography}{x}
\addcontentsline{toc}{chapter}{Referencias} % si queremos que aparezca en el índice

\bibitem[A, 1] {1} Bote, V. P. G., \& López-Pujalte, C. (2001). \textit{Inteligencia artificial y documentación. Investigación Bibliotecológica}, 15(30), 66.

\bibitem[A, 2] {2} Ponce, J. C., Torres, A., Quezada, F. S., Silva, A., Martínez, E. U., Casali, A., Scheihing, E., Túpac, Y. J., Torres, M. D., Ornelas, F. J., Hernández, J. A., Zavala, C., Vakhnia, N. \& Pedreño, O. (2014). \textit{Inteligencia Artificial}. España: Iniciativa Latinoamericana de Libros de Texto Abiertos (LATIn).

\bibitem[A, 3] {3} Fundación UNAM. (2014). \textit{Inteligencia Artificial en la medicina}. 2014, de Fundación UNAM Sitio web: http://www.fundacionunam.org.mx/mi-tecnologia/inteligencia-artificial-en-la-medicina.

\bibitem[A, 4] {4} Ponce, P. (2010). \textit{Inteligencia artificial con aplicaciones a la ingeniería}. México: Alfaomega.

\bibitem[A, 5] {5} Maynez, N. (2017). \textit{10 Tecnologías de Inteligencia Artificial que dominarán el 2018}. 2017, de ADEXT Sitio web: https://blog.adext.com/es/tecnologias-inteligencia-artificial-2017

\bibitem[A, 6] {6} Benítez, R., Escudero, G., Kanaan, S., \& Rodó, D. M. (2014). \textit{Inteligencia artificial avanzada}. Editorial UOC.

\bibitem[A, 7] {7} Haugeland, J. (1989). \textit{Artificial intelligence: The very idea}. MIT press, Cambridge, Massachusetts.

\bibitem[A, 8] {8} Bellman, R. (1978). \textit{An introduction to artificial intelligence: Can computers think?}. Thomson Course Technology.

\bibitem[A, 9] {9} Kurzweil, R., Richter, R., Kurzweil, R., \& Schneider, M. L. (1990). \textit{The age of intelligent machines}. Cambridge: MIT press.

\bibitem[A, 10] {10} Rich, E. \& Knight, K. (1991). \textit{Artificial intelligence}. McGraw-Hill.

\bibitem[A, 11] {11} Norvig, P., \& Russell, S. (2004). \textit{Inteligencia artificial}. Editora Campus.

\bibitem[A, 12] {12} Gardner, H. (1982). \textit{The Mind's New Science}. Basic Books, New York.

\bibitem[A, 13] {13} Frege, G. (1879). \textit{Conceptografía, un lenguaje de fórmulas, semejante al de la aritmética, para el pensamiento puro}. México: UNAM, Instituto de Investigaciones Filosóficas.

\bibitem[A, 14] {14} Lucas, J. R. (1961). \textit{Minds, machines and Gödel}. Philosophy, 36(137), 112-127

\bibitem[A, 15] {15} Penrose, R. (1989). \textit{The emperor's new mind: Concerning computers, minds, and the laws of physics}. Oxford Paperbacks.

\bibitem[A, 16] {16} Penrose, R. (1994). \textit{Shadows of the Mind} (Vol. 4). Oxford: Oxford University Press.

\bibitem[A, 17] {17} McCulloch, W. \& Pitts, W. (1943). \textit{A logical calculus of ideas immanent in nervous activity}. Bulletin of Mathematical Biophysics, 5: 115-133.

\bibitem[A, 18] {18} Rosenblatt, F. (1958). \textit{The perceptron: A probabilistic model for information storage and organization in the brain}. Psychological Review, 65(6): 386-408.

\bibitem[A, 19] {19} Wiener, N. (1948). \textit{Cybernetics}. Scientific American, 179(5), 14-19.

\bibitem[A, 20] {20} Chomsky, N. (1965). \textit{Aspects of the Theory of Syntax}. 16-75.

\bibitem[A, 21] {21} Newell, A., Shaw, J. C., \& Simon, H. A. (1959). \textit{A general problem-solving program for a computer}. Computers and Automation, 8(7), 10-16.

\bibitem[A, 22] {23} Newell, A. (1963). \textit{A guide to the general problem-solver program GPS-2-2}. RAND Corporation, Santa Monica, California. Technical Report No. RM-3337-PR.

\bibitem[A, 24] {24} Feigenbaum, E. A., \& Feldman, J. (1963). \textit{Computers and thought}. New York.

\bibitem[A, 25] {25} Buchanan, B. G., Feigenbaum, E. A., \& Lederberg, J. (1971). \textit{A heuristic programming study of theory formation in science}. Computer Science Department.

\bibitem[A, 26] {26} Shortliffe, E. H. (1976). \textit{Computer-based medical consultation}. MYCIN.

\bibitem[A, 27] {27} Miller, R. A., Pople Jr, H. E., \& Myers, J. D. (1982). \textit{Internist-I, an experimental computer-based diagnostic consultant for general internal medicine}. New England Journal of Medicine, 307(8), 468-476.

\bibitem[A, 28] {28} McDermott, J. (1982). \textit{R1: A rule-based configurer of computer systems}. Artificial intelligence, 19(1), 39-88.

\bibitem[A, 29] {29} Etzioni, O., \& Weld, D. (1994).\textit{ A softbot-based interface to the internet}. Communications of the ACM, 37(7), 72-76.


\end{thebibliography}